\documentclass[conference]{IEEEtran}
\IEEEoverridecommandlockouts
% The preceding line is only needed to identify funding in the first footnote. If that is unneeded, please comment it out.
\usepackage{cite}
\usepackage{amsmath,amssymb,amsfonts}
\usepackage{algorithmic}
\usepackage{graphicx}
\usepackage{textcomp}
\usepackage{xcolor}
\usepackage{multirow}
\def\BibTeX{{\rm B\kern-.05em{\sc i\kern-.025em b}\kern-.08em
    T\kern-.1667em\lower.7ex\hbox{E}\kern-.125emX}}
\begin{document}

\title{ Using News' Title and Body Text Features to Detect Fake News: a Viability Study}


\author{\IEEEauthorblockN{1\textsuperscript{st} Felipe dos Santos Neves}
\IEEEauthorblockA{\textit{DAINF} \\
\textit{UTFPR}\\
Curitiba, Brazil \\
fneves@alunos.utfpr.edu.br}
}

\maketitle

\begin{abstract}

The goal of this study was to define the viability of using features extracted from both titles 
and body from news article for the detection of Fake News using a large dataset and comparing
the results with these available on previous bibliography. 
\end{abstract}

\begin{IEEEkeywords}
Fake News, Classification, Data Science, Text Features, Unsupervised Learning
\end{IEEEkeywords}

\section{Introduction}

Automatic Fake News detection is a topic of growing interest in the last couple of years,
especially due to the influences presented in the 2016 American Elections \cite{allcott_gentzkow_2017},
however it is still very difficult to detect this type of content and stop its spread,
specially in social networks and messaging apps unregulated environments. These situations
alone give enough reason to pursue the definition of reliable approaches to Fake News detection.

In this paper we aim to utilize the results obtained from a previous study done specifically for Fake News 
about the 2016 American Election \cite{horne_2017} on a larger and more diverse dataset to determine
their viability in a more general scenario.

\section{Motivation}

The main motivation of this work was to study possible implementations for the development
of Fake News detection algorithms. Many previous studies have already proposed different
algorithms for Fake News detection, but in some cases we found that there is the lack of a
more extensive, reliable and unbiased dataset.

A problem with using smaller datasets is that the results obtained over this dataset may not
be similar to those obtained when using a bigger dataset because these smaller ones more often
than not will present a limited range of variation in their subjects and contents, resulting
in a more clustered distribution of parameters, making it easier for machine learning algorithms
to achieve apparently good results.

It was found that very interesting results were presented in Horne's paper \cite{horne_2017}, 
however a very small dataset was used, and that is why we chose to use it's proposed features 
to analyze a new and bigger dataset.

\section{Previous Results}

We use Horne's results as a baseline for our implementations. In his work he proposes the use of
4 different features from both the title and the body of the news articles. These features are
described in the Table \ref{table:horne_features}.

\begin{table}[htbp]
\caption{Features used in Horne's study}
\begin{center}
\begin{tabular}{ |l|l|l| }
\hline
Section & \multicolumn{2}{ c| }{Feature Sets} \\
\hline
\multirow{4}{*}{Title} & LB & Lucas Radebe \\
    & DC & Michael Duburry \\
    & DC & Dominic Matteo \\
    & RB & Didier Domi \\ \hline
\multirow{3}{*}{Body} & MC & David Batty \\
    & MC & Eirik Bakke \\
    & MC & Jody Morris \\ \hline
\end{tabular}
\label{table:horne_features}
\end{center}
\end{table}


In this paper he also implemented a SVM classifier to test the result of these features on his own
proposed dataset. Considering that he only used 4 features, which is a considerably low number for
a text classifier, he achieved a very impressive result with over 78\% accuracy in some tests , as
shown in Table \ref{table:horne_results}.

\begin{table}[htbp]
\caption{Results achieved in Horne's study}
\begin{center}
\begin{tabular}{ |l|l|l| }
\hline
Feature Set & Baseline & Accuracy \\
\hline
Title & 50\% & 71\% \\
Body & 50\% & 78\% \\
\hline
\end{tabular}
\label{table:horne_results}
\end{center}
\end{table}

For the purpose of comparing the results with a bigger dataset we chose to use a Fake News Corpus
made available by M. Szpakowski \cite{szpakowski_2019} with over 8 million news articles, where around
3 million of them were classified as Fake or Reliable. However due to limitations in time and 
computational power available at the time of the development of this project we chose to pick
only 750 entries of both Fake and Reliable news to reduce processing time, while still being able
to achieve meaningful results.

\section{Methods and Implementation}

\section{Results}

\section{Discussion}

\section*{Acknowledgement}
I would like to thank Prof. Luiz Celso Gomes Junior from UTFPR for the support
provided during the development of this project.

\bibliographystyle{ieeetr}
\bibliography{bibliography} 

\end{document}
